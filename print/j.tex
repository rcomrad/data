\documentclass{article}
\usepackage{lmodern}
\usepackage[utf8]{inputenc}
\usepackage[T1,T2A]{fontenc}

\usepackage{expl3}
\usepackage{CJKutf8}
\usepackage[english,russian]{babel}

\usepackage{multirow}
\usepackage{array}
\newcolumntype{P}[1]{>{\centering\arraybackslash}p{#1}}

\usepackage{geometry}
 \geometry{
 a4paper,
 %total={170mm,257mm},
 left=7mm,
 top=10mm,
 }
 
\begin{document}
\newgeometry{
 a4paper,
 left=7mm,
 top=30mm,
 }


\thispagestyle{empty}
\begin{center}

%\smallskip
%\vspace{200mm} %5mm vertical space
 {\huge МИНИСТЕРСТВО ПРОСВЯЩЕНИЯ }
\\
{\huge  РОССИЙСКОЙ ФЕДЕРАЦИИ }

%\bigskip
\vspace{50mm} %5mm vertical space
{\huge  ЖУРНАЛ }
\vspace{30mm} %5mm vertical space

{\huge УЧЁТА РАБОТЫ ПЕДАГОГА }
\\
\vspace{2mm} %5mm vertical space
{\huge ДОПОЛНИТЕЛЬНОГО ОБРАЗОВАНИЯ }
\\
\vspace{2mm} %5mm vertical space
{\huge  В ОБЪЕДИНЕНИИ (секции, клубе, кружке) }
\\
\vspace{2mm} %5mm vertical space
{\huge  на 2023 - 2024 учебный год }

\vspace{50mm} %5mm vertical space
{\huge C++  }
\\
\vspace{2mm} %5mm vertical space
{\huge Аникина Августина  }

\end{center}

\clearpage
\restoregeometry\begin{tabular}{ |c|p{5cm}|*{15}{P{0.4cm}|}  }

	\multicolumn{17}{r}{УЧЁТ ПОСЕЩАЕМОСТИ И ВЫПОЛНЕНИЯ} 
	\\ \hline

	& &
	\multicolumn{7}{|l}{Месяц}
	&                        
	\multicolumn{8}{r|}{Сентябрь }
	\\ \cline{3-17}

	№ & Фамилия, имя обучающегося &
	\multicolumn{15}{|c|}{Дата}
	\\ \cline{3-17}

	& &
	1&2&3&4&5&6&7&8&9&10&11&12&13&14&15 
	\\ \hline
	1&Агафонова Валерия&&&&&&&&&&&&&&&
\\ \hline
2&Вдовин Иван&&&&&&&&&&&&&&&
\\ \hline
3&Егоров Тимофей&&&&&&&&&&&&&&&
\\ \hline
4&Золотарева София&&&&&&&&&&&&&&&
\\ \hline
5&Ильин Матвей&&&&&&&&&&&&&&&
\\ \hline
6&Лебедева Стефания&&&&&&&&&&&&&&&
\\ \hline
7&Никитин Фёдор&&&&&&&&&&&&&&&
\\ \hline
8&Попов Камиль&&&&&&&&&&&&&&&
\\ \hline
9&Соколова Алиса&&&&&&&&&&&&&&&
\\ \hline
10&Субботина Полина&&&&&&&&&&&&&&&
\\ \hline
11& &&&&&&&&&&&&&&&
\\ \hline
12&  &&&&&&&&&&&&&&&
\\ \hline
13&   &&&&&&&&&&&&&&&
\\ \hline
14&    &&&&&&&&&&&&&&&
\\ \hline
15&     &&&&&&&&&&&&&&&
\\ \hline
16&      &&&&&&&&&&&&&&&
\\ \hline
17&       &&&&&&&&&&&&&&&
\\ \hline
18&        &&&&&&&&&&&&&&&
\\ \hline
19&         &&&&&&&&&&&&&&&
\\ \hline
20&          &&&&&&&&&&&&&&&
\\ \hline
21&           &&&&&&&&&&&&&&&
\\ \hline
22&            &&&&&&&&&&&&&&&
\\ \hline
23&             &&&&&&&&&&&&&&&
\\ \hline
24&              &&&&&&&&&&&&&&&
\\ \hline
25&               &&&&&&&&&&&&&&&
\\ \hline
26&                &&&&&&&&&&&&&&&
\\ \hline
27&                 &&&&&&&&&&&&&&&
\\ \hline
28&                  &&&&&&&&&&&&&&&
\\ \hline
29&                   &&&&&&&&&&&&&&&
\\ \hline
30&                    &&&&&&&&&&&&&&&
\\ \hline
 

\end{tabular}
\clearpage
\begin{tabular}{ |P{1.5cm}|p{7cm}|c|P{2cm}|p{2cm}|}  

	\multicolumn{5}{l}{ДОПОЛНИТЕЛЬНОЙ ОБРАЗОВАТЕЛЬНОЙ ПРОГРАММЫ (ДОП)} 
	\\ \hline

Даты занятий объединения & 
\multicolumn{1}{P{5cm}|}{Содержание занятий согласно ДОП } 

& Часы & Педагог &
\multicolumn{1}{P{5cm}|}{Правки} 
\\ \hline
1.09&Техника безопасности на рабочем месте&2&Аникина&
\\ \hline
2.09&Введение: Общие сведения о языке С++&2&Аникина&
\\ \hline
3.09&Введение: Работа со средой&2&Аникина&
\\ \hline
4.09&Первая программа&2&Аникина&
\\ \hline
5.09&Булева алгебра&2&Аникина&
\\ \hline
6.09&Общая структура программы на языке С++&2&Аникина&
\\ \hline
7.09&Типы данных&2&Аникина&
\\ \hline
8.09&Задача. Решение квадратного уравнения &2&Аникина&
\\ \hline
9.09&Математические функции&2&Аникина&
\\ \hline
10.09&Цикл while&2&Аникина&
\\ \hline
11.09&Цикл for&2&Аникина&
\\ \hline
12.09&Ввод-вывод данных&2&Аникина&
\\ \hline
13.09&Файловый ввод-вывод&2&Аникина&
\\ \hline
14.09&Задача на поиск суммы цифр числа&2&Аникина&
\\ \hline
15.09&Отоладка&2&Аникина&
\\ \hline

\end{tabular}
\clearpage
\begin{tabular}{ |c|p{5cm}|*{15}{P{0.4cm}|}  }

	\multicolumn{17}{r}{УЧЁТ ПОСЕЩАЕМОСТИ И ВЫПОЛНЕНИЯ} 
	\\ \hline

	& &
	\multicolumn{7}{|l}{Месяц}
	&                        
	\multicolumn{8}{r|}{Сентябрь }
	\\ \cline{3-17}

	№ & Фамилия, имя обучающегося &
	\multicolumn{15}{|c|}{Дата}
	\\ \cline{3-17}

	& &
	16&17&18&19&20&21&22&23&24&25&26&27&28&29&30 
	\\ \hline
	1&Агафонова Валерия&&&&&&&&&&&&&&&
\\ \hline
2&Вдовин Иван&&&&&&&&&&&&&&&
\\ \hline
3&Егоров Тимофей&&&&&&&&&&&&&&&
\\ \hline
4&Золотарева София&&&&&&&&&&&&&&&
\\ \hline
5&Ильин Матвей&&&&&&&&&&&&&&&
\\ \hline
6&Лебедева Стефания&&&&&&&&&&&&&&&
\\ \hline
7&Никитин Фёдор&&&&&&&&&&&&&&&
\\ \hline
8&Попов Камиль&&&&&&&&&&&&&&&
\\ \hline
9&Соколова Алиса&&&&&&&&&&&&&&&
\\ \hline
10&Субботина Полина&&&&&&&&&&&&&&&
\\ \hline
11& &&&&&&&&&&&&&&&
\\ \hline
12&  &&&&&&&&&&&&&&&
\\ \hline
13&   &&&&&&&&&&&&&&&
\\ \hline
14&    &&&&&&&&&&&&&&&
\\ \hline
15&     &&&&&&&&&&&&&&&
\\ \hline
16&      &&&&&&&&&&&&&&&
\\ \hline
17&       &&&&&&&&&&&&&&&
\\ \hline
18&        &&&&&&&&&&&&&&&
\\ \hline
19&         &&&&&&&&&&&&&&&
\\ \hline
20&          &&&&&&&&&&&&&&&
\\ \hline
21&           &&&&&&&&&&&&&&&
\\ \hline
22&            &&&&&&&&&&&&&&&
\\ \hline
23&             &&&&&&&&&&&&&&&
\\ \hline
24&              &&&&&&&&&&&&&&&
\\ \hline
25&               &&&&&&&&&&&&&&&
\\ \hline
26&                &&&&&&&&&&&&&&&
\\ \hline
27&                 &&&&&&&&&&&&&&&
\\ \hline
28&                  &&&&&&&&&&&&&&&
\\ \hline
29&                   &&&&&&&&&&&&&&&
\\ \hline
30&                    &&&&&&&&&&&&&&&
\\ \hline
 

\end{tabular}
\clearpage
\begin{tabular}{ |P{1.5cm}|p{7cm}|c|P{2cm}|p{2cm}|}  

	\multicolumn{5}{l}{ДОПОЛНИТЕЛЬНОЙ ОБРАЗОВАТЕЛЬНОЙ ПРОГРАММЫ (ДОП)} 
	\\ \hline

Даты занятий объединения & 
\multicolumn{1}{P{5cm}|}{Содержание занятий согласно ДОП } 

& Часы & Педагог &
\multicolumn{1}{P{5cm}|}{Правки} 
\\ \hline
16.09&Тестирование&2&Аникина&
\\ \hline
17.09&Оценка асимптотики программы&2&Аникина&
\\ \hline
18.09&Способы организации кода&2&Аникина&
\\ \hline
19.09&Поиск суммы элементов массива&2&Аникина&
\\ \hline
20.09&Понятие массива&2&Аникина&
\\ \hline
21.09&Целочисленные массивы&2&Аникина&
\\ \hline
22.09&Сумма элементов массива&2&Аникина&
\\ \hline
23.09&Поиск минимального и максимального элемента в массиве&2&Аникина&
\\ \hline
24.09&Подсчёт количества элементов с заданными свойствами в массиве&2&Аникина&
\\ \hline
25.09&Кодовая таблица ASCII&2&Аникина&
\\ \hline
26.09&Палиндромы&2&Аникина&
\\ \hline
27.09&Контейнер vector&2&Аникина&
\\ \hline
28.09&Некоторые функции библиотеки algorithm&2&Аникина&
\\ \hline
29.09&Понятие двумерных массивов&2&Аникина&
\\ \hline
30.09&Задачи на вывод двумерных массивов, таблиц&2&Аникина&
\\ \hline

\end{tabular}
\clearpage
\begin{tabular}{ |c|p{5cm}|*{15}{P{0.4cm}|}  }

	\multicolumn{17}{r}{УЧЁТ ПОСЕЩАЕМОСТИ И ВЫПОЛНЕНИЯ} 
	\\ \hline

	& &
	\multicolumn{7}{|l}{Месяц}
	&                        
	\multicolumn{8}{r|}{Октябрь }
	\\ \cline{3-17}

	№ & Фамилия, имя обучающегося &
	\multicolumn{15}{|c|}{Дата}
	\\ \cline{3-17}

	& &
	1&2&3&4&5&6&7&8&9&10&11&12&13&14&15 
	\\ \hline
	1&Агафонова Валерия&&&&&&&&&&&&&&&
\\ \hline
2&Вдовин Иван&&&&&&&&&&&&&&&
\\ \hline
3&Егоров Тимофей&&&&&&&&&&&&&&&
\\ \hline
4&Золотарева София&&&&&&&&&&&&&&&
\\ \hline
5&Ильин Матвей&&&&&&&&&&&&&&&
\\ \hline
6&Лебедева Стефания&&&&&&&&&&&&&&&
\\ \hline
7&Никитин Фёдор&&&&&&&&&&&&&&&
\\ \hline
8&Попов Камиль&&&&&&&&&&&&&&&
\\ \hline
9&Соколова Алиса&&&&&&&&&&&&&&&
\\ \hline
10&Субботина Полина&&&&&&&&&&&&&&&
\\ \hline
11& &&&&&&&&&&&&&&&
\\ \hline
12&  &&&&&&&&&&&&&&&
\\ \hline
13&   &&&&&&&&&&&&&&&
\\ \hline
14&    &&&&&&&&&&&&&&&
\\ \hline
15&     &&&&&&&&&&&&&&&
\\ \hline
16&      &&&&&&&&&&&&&&&
\\ \hline
17&       &&&&&&&&&&&&&&&
\\ \hline
18&        &&&&&&&&&&&&&&&
\\ \hline
19&         &&&&&&&&&&&&&&&
\\ \hline
20&          &&&&&&&&&&&&&&&
\\ \hline
21&           &&&&&&&&&&&&&&&
\\ \hline
22&            &&&&&&&&&&&&&&&
\\ \hline
23&             &&&&&&&&&&&&&&&
\\ \hline
24&              &&&&&&&&&&&&&&&
\\ \hline
25&               &&&&&&&&&&&&&&&
\\ \hline
26&                &&&&&&&&&&&&&&&
\\ \hline
27&                 &&&&&&&&&&&&&&&
\\ \hline
28&                  &&&&&&&&&&&&&&&
\\ \hline
29&                   &&&&&&&&&&&&&&&
\\ \hline
30&                    &&&&&&&&&&&&&&&
\\ \hline
 

\end{tabular}
\clearpage
\begin{tabular}{ |P{1.5cm}|p{7cm}|c|P{2cm}|p{2cm}|}  

	\multicolumn{5}{l}{ДОПОЛНИТЕЛЬНОЙ ОБРАЗОВАТЕЛЬНОЙ ПРОГРАММЫ (ДОП)} 
	\\ \hline

Даты занятий объединения & 
\multicolumn{1}{P{5cm}|}{Содержание занятий согласно ДОП } 

& Часы & Педагог &
\multicolumn{1}{P{5cm}|}{Правки} 
\\ \hline
1.10&Задачи на тему игры морской бой&2&Аникина&
\\ \hline
2.10&Дискретная математика&2&Аникина&
\\ \hline
3.10&Понятие графа&2&Аникина&
\\ \hline
4.10&Представление графа в программе&2&Аникина&
\\ \hline
5.10&Ввод-вывод графа&2&Аникина&
\\ \hline
6.10&Алгоритм BFS&2&Аникина&
\\ \hline
7.10&Примитивная реализация алгоритма BFS&2&Аникина&
\\ \hline
8.10&Задачи на обход в ширину&2&Аникина&
\\ \hline
9.10&Понятие динамического программирования&2&Аникина&
\\ \hline
10.10&Задачи на поиск максимальной возрастающей подпоследовательности&2&Аникина&
\\ \hline
11.10&Задачи регионального поиска &2&Аникина&
\\ \hline
12.10&Задачи на теорию автоматов&2&Аникина&
\\ \hline
13.10&Жадные алгоритмы&2&Аникина&
\\ \hline
14.10&Простые числа, решето Эратосфена &2&Аникина&
\\ \hline
15.10&Алгоритм Евклида для нахождения НОД и его применение в задачах&2&Аникина&
\\ \hline

\end{tabular}
\clearpage
\begin{tabular}{ |c|p{5cm}|*{15}{P{0.4cm}|}  }

	\multicolumn{17}{r}{УЧЁТ ПОСЕЩАЕМОСТИ И ВЫПОЛНЕНИЯ} 
	\\ \hline

	& &
	\multicolumn{7}{|l}{Месяц}
	&                        
	\multicolumn{8}{r|}{Октябрь }
	\\ \cline{3-17}

	№ & Фамилия, имя обучающегося &
	\multicolumn{15}{|c|}{Дата}
	\\ \cline{3-17}

	& &
	16&17&18&19 
	\\ \hline
	1&Агафонова Валерия&&&&&&&&&&&&&&&
\\ \hline
2&Вдовин Иван&&&&&&&&&&&&&&&
\\ \hline
3&Егоров Тимофей&&&&&&&&&&&&&&&
\\ \hline
4&Золотарева София&&&&&&&&&&&&&&&
\\ \hline
5&Ильин Матвей&&&&&&&&&&&&&&&
\\ \hline
6&Лебедева Стефания&&&&&&&&&&&&&&&
\\ \hline
7&Никитин Фёдор&&&&&&&&&&&&&&&
\\ \hline
8&Попов Камиль&&&&&&&&&&&&&&&
\\ \hline
9&Соколова Алиса&&&&&&&&&&&&&&&
\\ \hline
10&Субботина Полина&&&&&&&&&&&&&&&
\\ \hline
11& &&&&&&&&&&&&&&&
\\ \hline
12&  &&&&&&&&&&&&&&&
\\ \hline
13&   &&&&&&&&&&&&&&&
\\ \hline
14&    &&&&&&&&&&&&&&&
\\ \hline
15&     &&&&&&&&&&&&&&&
\\ \hline
16&      &&&&&&&&&&&&&&&
\\ \hline
17&       &&&&&&&&&&&&&&&
\\ \hline
18&        &&&&&&&&&&&&&&&
\\ \hline
19&         &&&&&&&&&&&&&&&
\\ \hline
20&          &&&&&&&&&&&&&&&
\\ \hline
21&           &&&&&&&&&&&&&&&
\\ \hline
22&            &&&&&&&&&&&&&&&
\\ \hline
23&             &&&&&&&&&&&&&&&
\\ \hline
24&              &&&&&&&&&&&&&&&
\\ \hline
25&               &&&&&&&&&&&&&&&
\\ \hline
26&                &&&&&&&&&&&&&&&
\\ \hline
27&                 &&&&&&&&&&&&&&&
\\ \hline
28&                  &&&&&&&&&&&&&&&
\\ \hline
29&                   &&&&&&&&&&&&&&&
\\ \hline
30&                    &&&&&&&&&&&&&&&
\\ \hline
 

\end{tabular}
\clearpage
\begin{tabular}{ |P{1.5cm}|p{7cm}|c|P{2cm}|p{2cm}|}  

	\multicolumn{5}{l}{ДОПОЛНИТЕЛЬНОЙ ОБРАЗОВАТЕЛЬНОЙ ПРОГРАММЫ (ДОП)} 
	\\ \hline

Даты занятий объединения & 
\multicolumn{1}{P{5cm}|}{Содержание занятий согласно ДОП } 

& Часы & Педагог &
\multicolumn{1}{P{5cm}|}{Правки} 
\\ \hline
16.10&Сортировка подсчётом &2&Аникина&
\\ \hline
17.10&Топологическая сортировка&2&Аникина&
\\ \hline
18.10&Быстрая сортировка&2&Аникина&
\\ \hline
19.10&Блочная сортировка&2&Аникина&
\\ \hline

\end{tabular}
\clearpage
\end{document}